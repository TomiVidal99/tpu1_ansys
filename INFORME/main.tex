%%%%%%%%%%%%%%%%%%%%%%%%%%%%%%%%%%%%%%%%%%%%%%%%%%%%%%%%%%%%%%%%%%%%%%%%%%%%%%%%
%2345678901234567890123456789012345678901234567890123456789012345678901234567890
%        1         2         3         4         5         6         7         8
\documentclass[letterpaper, 10 pt, conference]{ieeeconf}  % Comment this line out
                                                          % if you need a4paper
%\documentclass[a4paper, 10pt, conference]{ieeeconf}      % Use this line for a4

\usepackage{float}
                                                          % paper
% uso paquete bookmark para tener bien los outlines.
\usepackage{bookmark}

% Configuro el idioma.
\usepackage[utf8]{inputenc} % Importante para mantener acentos.
\usepackage[spanish, activeacute]{babel} % Requiere: texlive-lang-spanish. Por primera vez hay que ejecutar: texconfig init> log

% Paquete para poder usar acentos en $$.
\usepackage{mathtools}
%\setmathfont{XITS math}

% package to get \url
\usepackage{hyperref}
\hypersetup{
  colorlinks=true,
  linkcolor=magenta,
  filecolor=magenta,
  citecolor=magenta,      
  urlcolor=magenta,
}

\IEEEoverridecommandlockouts                              % This command is only
                                                          % needed if you want to
                                                          % use the \thanks command
\overrideIEEEmargins
% See the \addtolength command later in the file to balance the column lengths
% on the last page of the document

\usepackage{graphicx}
\usepackage{graphics}

% styling for matlab/octave code.
\usepackage{matlab-prettifier}
% Configuracion, con esto puede agregar ñ.
\lstset{
  literate={ñ}{{\~n}}1
}

% The following packages can be found on http:\\www.ctan.org
%\usepackage{graphics} % for pdf, bitmapped graphics files
%\usepackage{epsfig} % for postscript graphics files
%\usepackage{mathptmx} % assumes new font selection scheme installed
%\usepackage{times} % assumes new font selection scheme installed
\usepackage{amsmath} % assumes amsmath package installed
%\usepackage{amssymb}  % assumes amsmath package installed

\title{\LARGE \bf Pr\'actica con Utilitario N° 1}

\author{
  Tom\'as Vidal\\
  {\it An\'alisis de Sistemas y Se\~{n}ales}\\
  {\it Facultad de Ingenier\'ia, UNLP, La Plata, Argentina.}\\
  {\it 21 de Mayo, 2023.}
}
% <-this % stops a space


% comienzo

% INTRO


% Figura
\newcommand{\image}[2] {
  \begin{figure}[H]
    \centering
    \includegraphics[width=0.43\textwidth]{../figures/#1.png}
    \caption{#2}
    \label{fig:#1}
  \end{figure}
}

% Codigo
% \begin{lstlisting}[style=Matlab-editor]
% % el código va aca
% dispc("HELLO WORLD");
% \end{lstlisting}

\begin{document}
\maketitle
\thispagestyle{empty}
\pagestyle{empty}

\section{INTRODUCCCI\'ON}
En el presente se desarrollan detalladamente los m\'etodos y t\'ecnicas empleadas para la resoluci\'on de sistemas discretos, en particular de tipo lineal e invariante al desplazamiento (o en el tiempo), haciendo uso de la herramienta \href{https://octave.org/}{Octave}. \'Estos sistemas est\'an descriptos en ecuaciones en diferencias\footnote{ver ASyS2023\_PU1-1.pdf.}.
% Se resulve el trabajo pr\'actico dado a partir de los conocimientos y algoritmos aprendidos en las clases de Análisis de Sistemas y Se\~{n}ales. Para lo cual se cre\'o un script (main.m) que ejecuta todos los comandos y alogritmos pertinentes para generar los datos y gr\'aficos que se muestran posteriormente. Adem\'as todo el trabajo se encuentra hosteado en un \href{https://github.com/TomiVidal99/tpu1_ansys}{repositorio}\footnote{El repositorio se encuentra privado, para acceder hay que enviar un email a \href{mailto:vidal.tomas@alu.ing.unlp.edu.ar}{vidal.tomas@alu.ing.unlp.edu.ar}} de Github\footnote{Servicio basado en la nube que ayuda a desarrolladores a guardar y gestinonar c\'odigo.} por si el lector quiere indagar profundamente en el c\'odigo y en los algoritmos.

\section{Implementaci\'on de la TFTD}
\subsection{Marco Te\'orico} \label{subsec:TFTD}
La transformada de Fourier de tiempo discreto (TFTD) de una se\~{n}al dada se define como (Ref. \cite{tftd_tp5} y \cite{tftd_teoria})
\[
  TFTD\{x[n]\} = X(e^{j2{\pi}s}) = \sum_{-\infty}^{\infty}{x[n]e^{-j2{\pi}sn}}
\]
Pero como no se pueden computar infinitos t\'erminos se debe hacer una aproximaci\'on de la siguiente manera
\[
  \hat{X}(e^{j2{\pi}s}) \approx \sum_{n=-K}^{K}{x[n]e^{-j2{\pi}sn}}
\]
Aproximar desde $n=-K$ hasta $n=K$ es lo mismo que multiplicar toda la se\~nal por un caj\'on y sumarla desde $-\infty$ hasta $\infty$, es decir que el efecto de aproximar es despreciar informaci\'on de la se\~nal.
\[
  \sum_{n=-K}^{K}{x[n]e^{-j2{\pi}sn}} = \sum_{n=-\infty}^{\infty}{x[n]\sqcap[\frac{n}{2K}]e^{-j2{\pi}sn}}
\]

\subsection{Algoritmo en Octave}
\begin{lstlisting}[style=Matlab-editor]
function [s, tftd] = TFTD(
  n, signal,
  defaultStep=1e-3)
  ds = defaultStep;
  s = [-.5:ds:.5];
  tftd = zeros(size(s));
  for k = 1:length(s)
    tftd(k) = sum(signal.*exp(-1j*2*pi*s(k)*n));
  end
end
\end{lstlisting}

\subsection{Prueba del Algoritmo}
Para poder hacer uso del algoritmo primero se realiz\'o el c\'alculo analítico de la TFTD de las siguientes se\~{n}ales
\[
  Delta\ desplazada = \delta[n-n_o] \supset e^{-j2{\pi}sn_o}
\]
\[
  Tri\acute{a}ngulo = \wedge[n] \supset sinc^2[n]
\]
Y luego se emple\'o el siguiente c\'odigo para verificar que las aproximaciones sean lo ``suficientemente`` buenas

\begin{lstlisting}[style=Matlab-editor]
N = [1:100];

% Triangulo desplazado 50 y de base 5.
Senial1 = tri((N-length(N)/2)/5);
[s1, tftd1] = TFTD(N, Senial1);

% Delta desplazada 20.
Senial2 = zeros(size(N));
Senial2(20) = 1;
[s2, tftd2] = TFTD(N, Senial2);
\end{lstlisting}

De lo cual se obtuvieron los siguientes resultados (Fig. \ref{fig:triangulo_tftd} y \ref{fig:delta_tftd}), que son acorde a los c\'alculos analíticos, por lo que se concluye que el algoritmo implementado es adecuado para esta aplicaci\'on.

% Resultados gráficos
\image{triangulo_tftd}{Plot de la Se\~{n}al tri\'angulo y el m\'odulo y fase de su TFTD.}
\image{delta_tftd}{Plot de la Se\~{n}al delta y el m\'odulo y fase de su TFTD.}
% Entonces haciendo uso del algoritmo explicado previamente (ver \ref{commands_tftf_signal}) se aproxim\'o la TFTD de la se\~{n}al dada (ver \ref{fig:senial_y_su_tftd}), de lo cual se puede observar que la se\~{n}al est\'a compuesta por otras dos: una de baja frecuencia preponderante, y otra de baja que influye menos (ver Fig. \ref{fig:tftd_senial_freqs}).
% \image{senial_y_su_tftd}{Plot de la Se\~{n}al y el m\'odulo y fase de su TFTD.}
% Grafico de la frecuencia, a la que se le señala las altas y bajas frecuencias.
% \begin{figure}[H]
% \centering
% \includegraphics[width=0.43\textwidth]{figures/senial_y_su_tftd_indicating.png}
% \caption{Se indica las frecuencias predominantes en la TFTD de la se\~nal.}
% \label{fig:tftd_senial_freqs}
% \end{figure}
% \subsection{Comandos para la obtenci\'on de la se\~{n}al y su TFTD}
% \label{commands_tftf_signal}
% \begin{lstlisting}[style=Matlab-editor]
% [n, x] = senial(698544);
% [s, tftd] = TFTD(n, x, 1e-4);
% \end{lstlisting}


% \section{Implementaci\'on de TFTD en Octave}
% Se cre\'o un \textit{script}\footnote{El \textit{script} se puede encontrar como \textit{'TFTD.m'}} el cual calcula la TFTD\footnote{Transformada de Fourier de tiempo discreto} de una se\~nal. Para corroborar que funciona correctamente se calcularon las TFTD de los siguientes pares conocidos.
% \[
%   TFTD\{x[n]\} = X(e^{j2{\pi}s}) = \sum_{n=-\infty}^{\infty}{x[n]e^{-j2{\pi}sn}}
% \]
%
% \image{senial}{Se\~nal provista.}
%
% \section{Procesamiento de la se\~nal dada}
% Se obtuvo la se\~nal dada a partir de los comandos provistos\footnote{use el comando 'help senial' para m\'as informaci\'on.}, y como se puede apreciar en el gr\'afico (fig. \ref{fig:senial}) posee componentes de alta y baja frecuencia.
% \image{senial}{Se\~nal provista.}

% \subsection{}

% \subsection{}
% Analizando el diagrama en bloques provisto se obtiene la siguiente ecuaci\'on que describe su comportamiento.
% \begin{equation} \label{eq:y}
%   y[n] = \frac{1}{2}x[n] + \frac{1}{2}x[n-1] + R^{L}y[n-L]
% \end{equation}
% Se obtiene la transferencia del sistema transformando ec. \ref{eq:y} con la transformanda $Z$.
% \begin{equation} \label{eq:Hz}
%   H(z) = \frac{Y(z)}{X(z)} = (\frac{1}{2})\frac{1 + z^{-1}}{1 - R^{L}z^{-L}}
% \end{equation}
%
% \subsection{Obtenci\'on de la ecuaci\'on del sistema}
% Analizando el diagrama en bloques provisto se obtiene la siguiente ecuaci\'on que describe su comportamiento.
% \begin{equation} \label{eq:y}
%   y[n] = \frac{1}{2}x[n] + \frac{1}{2}x[n-1] + R^{L}y[n-L]
% \end{equation}
% Se obtiene la transferencia del sistema transformando ec. \ref{eq:y} con la transformanda $Z$.
% \begin{equation} \label{eq:Hz}
%   H(z) = \frac{Y(z)}{X(z)} = (\frac{1}{2})\frac{1 + z^{-1}}{1 - R^{L}z^{-L}}
% \end{equation}
%
% Mediante la funci\'on de transferencia se obtiene la respuesta en frecuencia (fig. \ref{fig:HfreqResp}) del sistema de la siguiente manera
% \begin{equation*}
%   H(e^{j2{\pi}s}) = (\frac{1}{2})\frac{1 + {e^{-j2{\pi}s}}}{1 - R^{L}{e^{-j2{\pi}sL}}}
% \end{equation*}
% % TODO hacer una animación para diferentes valores de L y R
% \image{HfreqResp}{Respuesta en frecuencia del sistema (ec. \ref{eq:y})}
%
% \subsection{Polos y ceros de H(z)}
% Se grafican los polos y los ceros de la funci\'on de transferencia (ec. \ref{eq:Hz}) con el siguiente c\'odigo
% % Codigo
% % TODO: actualizar esto con el correspondiente codigo
% \begin{lstlisting}[style=Matlab-editor]
% % tf y pzmap son del paquete de control
% pkg load control;
% % Se define el tipo de z
% z = tf('z', Ts);
% % Se define la funcion de transferencia H(z) como H
% H = @(r, l) (1/2)*( (1+z^(-1)) / (1-(r^l)*(z^(-l))) );
% % Se define una secuencia aleatoria uniforme de longitud 100
% x = rand(100);
% L = length(x);
% R = .1;
% % Se grafican los zeros y polos de H
% pzmap(H(R, L));
% \end{lstlisting}
%
% \image{Hz-pzmap-1}{Polos y ceros de la función de transferencia (ec. \ref{eq:Hz}), con R=10}
% \image{Hz-pzmap-2}{Polos y ceros de la función de transferencia (ec. \ref{eq:Hz}), con R=5}
% \image{Hz-pzmap-3}{Polos y ceros de la función de transferencia (ec. \ref{eq:Hz}), con R=1}
% \image{Hz-pzmap-4}{Polos y ceros de la función de transferencia (ec. \ref{eq:Hz}), con R=0,5}
% \image{Hz-pzmap-5}{Polos y ceros de la función de transferencia (ec. \ref{eq:Hz}), con R=0,1}
% \image{Hz-pzmap-6}{Polos y ceros de la función de transferencia (ec. \ref{eq:Hz}), con R=0,01}
%
% Haciendo un an\'alisis de la ec. \ref{eq:Hz} podemos observar que el numerador tiende a cero cuando $z$ tiende a $-1$, y el denominador cuando $z$ tiende a $R$; por lo que los polos se encuentran en la circunferencia de radio $R$ y son L (en cantidad), adem\'as tiene un cero en $z=-1$, que es acorde con lo que se ver en los gr\'aficos anteriores.
%
% \subsection{Estabilidad del sistema}
% Para que el sistema sea estable y causal todos los polos se deben encontrar dentro de la circunferencia de radio unitario. Por lo que para valores de $0<R<1$ el sistema es estable.
%
% \section{Inciso 2}
% \subsection{Filtro de pasatodo}
% Obtengo la ecuación (\ref{eq:yfiltro}) que describe el sistema a partir de los diagramas en bloques dados.
% \begin{equation} \label{eq:yfiltro}
%   y[n] = a.x[n] + x[n-1] - a.y[n-1]
% \end{equation}
% Calculo la función de transferencia a partir de la ecuación \ref{eq:yfiltro}.
% \begin{equation} \label{eq:Hfiltro}
%   H(z) = \frac{a + z^{-1}}{1+a.z^{-1}}
% \end{equation}
% \subsection{Fase del filtro}
% En el siguiente se muestran diferentes fases de la respuesta en frecuencia del filtro (ec \ref{eq:yfiltro}).
% \image{HPhiA}{Diferentes fases de la respuesta en frecuencia del filtro (ec. \ref{eq:yfiltro}).}

\begin{thebibliography}{99}

% TODO
\bibitem{tftd_tp5}ANÁLISIS DE SISTEMAS Y SE\~{n}ALES - A\~{n}O 2022, Práctica 5 Transformada de Fourier de Tiempo Discreto (TFTD), Serie Discreta Fourier (SDF).
\bibitem{tftd_teoria}ANÁLISIS DE SISTEMAS Y SE\~{n}ALES - A\~{n}O 2022, Filminas de teor\'ia 5: Transformada de Fourier de Tiempo Discreto (TFTD).

\end{thebibliography}

\end{document}
