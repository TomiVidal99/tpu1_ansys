%%%%%%%%%%%%%%%%%%%%%%%%%%%%%%%%%%%%%%%%%%%%%%%%%%%%%%%%%%%%%%%%%%%%%%%%%%%%%%%%
%2345678901234567890123456789012345678901234567890123456789012345678901234567890
%        1         2         3         4         5         6         7         8
\documentclass[letterpaper, 10 pt, conference]{ieeeconf}  % Comment this line out
                                                          % if you need a4paper
%\documentclass[a4paper, 10pt, conference]{ieeeconf}      % Use this line for a4

\usepackage{float}
                                                          % paper
% uso paquete bookmark para tener bien los outlines.
\usepackage{bookmark}

% Configuro el idioma.
\usepackage[utf8]{inputenc} % Importante para mantener acentos.
\usepackage[spanish, activeacute]{babel} % Requiere: texlive-lang-spanish. Por primera vez hay que ejecutar: texconfig init> log

% Paquete para poder usar acentos en $$.
\usepackage{mathtools}
%\setmathfont{XITS math}

% package to get \url
\usepackage{hyperref}
\hypersetup{
  colorlinks=true,
  linkcolor=magenta,
  filecolor=magenta,
  citecolor=magenta,      
  urlcolor=magenta,
}

\IEEEoverridecommandlockouts                              % This command is only
                                                          % needed if you want to
                                                          % use the \thanks command
\overrideIEEEmargins
% See the \addtolength command later in the file to balance the column lengths
% on the last page of the document

\usepackage{graphicx}
\usepackage{graphics}

% styling for matlab/octave code.
\usepackage{matlab-prettifier}
% Configuracion, con esto puede agregar ñ.
\lstset{
  literate={ñ}{{\~n}}1
}

% The following packages can be found on http:\\www.ctan.org
%\usepackage{graphics} % for pdf, bitmapped graphics files
%\usepackage{epsfig} % for postscript graphics files
%\usepackage{mathptmx} % assumes new font selection scheme installed
%\usepackage{times} % assumes new font selection scheme installed
\usepackage{amsmath} % assumes amsmath package installed
%\usepackage{amssymb}  % assumes amsmath package installed

\title{\LARGE \bf Pr\'actica con Utilitario N° 1}

\author{
  Tom\'as Vidal\\
  {\it An\'alisis de Sistemas y Se\~{n}ales}\\
  {\it Facultad de Ingenier\'ia, UNLP, La Plata, Argentina.}\\
  {\it 21 de Mayo, 2023.}
}
% <-this % stops a space


% comienzo

% INTRO


% Figura
\newcommand{\image}[2] {
  \begin{figure}[H]
    \centering
    \includegraphics[width=0.43\textwidth]{../figures/#1.png}
    \caption{#2}
    \label{fig:#1}
  \end{figure}
}

% Codigo
% \begin{lstlisting}[style=Matlab-editor]
% % el código va aca
% dispc("HELLO WORLD");
% \end{lstlisting}

\begin{document}
\maketitle
\thispagestyle{empty}
\pagestyle{empty}

\section{INTRODUCCCI\'ON}
En el presente se desarrollan detalladamente los \textbf{m\'etodos} y \textbf{t\'ecnicas} empleadas para la \textit{caracterizaci\'on y an\'alisis} de \textbf{sistemas discretos}, en particular de tipo lineal e invariante al desplazamiento (o en el tiempo), haciendo uso de la herramienta \href{https://octave.org/}{Octave}\footnote{Octave o GNU Octave es un programa y lenguaje de programaci\'on para realizar c\'alculos num\'ericos. Como su nombre indica, Octave es parte del proyecto GNU. Es considerado el equivalente libre de MATLAB}. \'Estos sistemas est\'an descriptos en ecuaciones en diferencias\footnote{ver ASyS2023\_PU1-1.pdf.}, y son implementados con algoritmos en \textbf{scripts}\footnote{En inform\'atica, un script, refiere a un archivo que contiene una secuencia de comandos de un lenguaje de programaci\'on}. Adem\'as se analizan y solucionan los problemas presentes en un canal de transmisi\'on cuya respuesta impulsional es dada como informaci\'on, para poder enviar por el mismo una se\~nal sin que se distorsione tanto.

\section{Implementaci\'on de la \textbf{TFTD} en Octave}
\subsection{Marco Te\'orico} \label{subsec:TFTD}
La \textbf{transformada de Fourier de tiempo discreto} (\textbf{TFTD}) de una se\~{n}al dada se define como (\cite{bib:tftd_tp5} y \cite{bib:tftd_teoria})
\[
  TFTD\{x[n]\} = X(e^{j2{\pi}s}) = \sum_{-\infty}^{\infty}{x[n]e^{-j2{\pi}sn}}
\]
Pero como no se pueden computar infinitos t\'erminos se debe hacer una aproximaci\'on de la siguiente manera
\[
  \hat{X}(e^{j2{\pi}s}) \approx \sum_{n=-K}^{K}{x[n]e^{-j2{\pi}sn}}
\]
Aproximar desde $n=-K$ hasta $n=K$ es lo mismo que multiplicar toda la se\~nal por un caj\'on y sumarla desde $-\infty$ hasta $\infty$, es decir que el efecto de aproximar es despreciar informaci\'on de la se\~nal.
\[
  \sum_{n=-K}^{K}{x[n]e^{-j2{\pi}sn}} = \sum_{n=-\infty}^{\infty}{x[n]\sqcap[\frac{n}{2K}]e^{-j2{\pi}sn}}
\]

\subsection{Algoritmo en Octave}
\label{tftd_function}
\nopagebreak
\begin{lstlisting}[style=Matlab-editor] 
function [s, tftd] = TFTD(
  n, signal,
  defaultStep=1e-3)
  ds = defaultStep;
  s = [-.5:ds:.5];
  tftd = zeros(size(s));
  for k = 1:length(s)
    tftd(k) = sum(signal.*exp(-1j*2*pi*s(k)*n));
  end
end
\end{lstlisting}

\subsection{Prueba del Algoritmo}
Para poder hacer uso del algoritmo primero se realiz\'o el c\'alculo analítico de la \textbf{TFTD} de las siguientes se\~{n}ales a partir de propiedades y pares transformandos conocidos
\[
  Delta = \delta[\textbf{n}-n_o] \supset e^{-j2{\pi}\textbf{s}n_o}
\]
\[
  Tri\acute{a}ngulo = \wedge[\frac{\textbf{n}-b}{a}] \supset |a|.e^{-j2{\pi}\textbf{s}\frac{a}{b}}.sinc^2(a\textbf{s})
\]
Y luego se emple\'o el siguiente c\'odigo que calcula la \textbf{TFTD} de manera num\'erica con el algoritmo (ver \ref{tftd_function}) y la \textbf{TFTD} anal\'itica, para verificar que las aproximaciones sean lo ``suficientemente'' buenas

\begin{lstlisting}[style=Matlab-editor]
N = [1:100];

% Delta desplazada 20.
Senial1 = zeros(size(N0));
Senial1(20) = 1;
[s1, tftd1] = TFTD(N0, Senial1);

% Triangulo desplazado 50 y de base 5.
Senial2 = tri((N0-length(N0)/2)/5);
[s2, tftd2] = TFTD(N0, Senial2);

% Transformada analitica senial1
tftdAnaliticSenial1 = @(s) exp(-j*2*pi*20.*s);

% Transformada analitica senial2
tftdAnaliticSenial2 = @(s) 5.*exp(-2*pi*10.*s).*(sinc(5.*s).^2);

\end{lstlisting}

De lo cual se obtuvieron los siguientes resultados (Fig. \ref{fig:delta_tftd} y \ref{fig:triangulo_tftd}), que son acordes por la similitud a los c\'alculos analíticos (Fig. \ref{fig:delta_tftd_analitica} y \ref{fig:triangulo_tftd_analitica} respectivamente), por lo que se concluye que el algoritmo implementado es adecuado para esta aplicaci\'on. Cabe aclarar que si bien hay variaciones en las fases, \'estas son muy susceptibles a cambios en las aproximaciones y, adem\'as no son de nuestro inter\'es principal, ya que se quieren analizar los m\'odulos de las \textbf{TFTD}.

% Resultados gráficos
\image{delta_tftd}{Se\~{n}al delta y el m\'odulo y fase de su TFTD num\'erica.}
\image{triangulo_tftd}{Se\~{n}al tri\'angulo y el m\'odulo y fase de su TFTD num\'erica.}
\image{delta_tftd_analitica}{Se\~{n}al delta y el m\'odulo y fase de su TFTD anal\'itica.}
\image{triangulo_tftd_analitica}{Se\~{n}al tri\'angulo y el m\'odulo y fase de su TFTD anal\'itica.}


\section{TFTD de la sen\~al provista}

Se obtuvo la se\~nal dada a partir de los comandos provistos\footnote{use el comando 'help senial' para m\'as informaci\'on.}, y como se puede apreciar en el gr\'afico (fig. \ref{fig:senial}) la misma est\'a compuesta de sinusoides superpuestas de componentes de alta y baja frecuencia.
\image{senial}{Se\~nal provista.}

Entonces haciendo uso del algoritmo desarrollado previamente (ver \ref{tftd_function}) se aproxim\'o la TFTD de la se\~{n}al dada (ver \ref{fig:senial_y_su_tftd}), de lo cual se puede observar que efectivamente lo hipotetizado sobre la se\~{n}al es cierto, pues la misma est\'a compuesta por dos componentes principales: una de baja frecuencia, preponderante, y otra de alta de menor peso (ver Fig. \ref{fig:tftd_senial_freqs}).
\image{senial_y_su_tftd}{Plot de la Se\~{n}al y el m\'odulo y fase de su TFTD.}
\image{tftd_senial_freqs}{Se indica las frecuencias predominantes en la TFTD de la se\~nal.}

% EJERCICIO 1 INCISO 2
\section{Sistemas dados}
% Sistemas de ecuaciones.
\subsection{Ecuaciones en diferencias}
Las siguientes ecuaciones en diferencias dadas representan sistemas
\begin{equation} \label{eq1}
y[n] = \frac{1}{2}x[n] + \frac{1}{2}x[n-1]
\end{equation}
\begin{equation} \label{eq2}
y[n] = \frac{1}{2}x[n] - \frac{1}{2}x[n-1]
\end{equation}
\begin{equation} \label{eq3}
y[n] = \frac{1}{4}x[n] + \frac{1}{4}x[n-1] + \frac{1}{2}y[n-1]
\end{equation}
\begin{equation} \label{eq4}
y[n] = \frac{1}{4}x[n] - \frac{1}{4}x[n-1] - \frac{1}{2}y[n-1]
\end{equation}

Las respuestas impulsionales de los sistemas \ref{eq1} y \ref{eq2} se obtienen anal\'iticamente cuando la entrada es $x[n] = \delta[n]$.
% Respuestas impusionales de los sistemas 1 y 2.
\begin{equation} \label{h1}
h_1[n] = \frac{1}{2}\delta[n] + \frac{1}{2}\delta[n-1]
\end{equation}
\begin{equation} \label{h2}
h_2[n] = \frac{1}{2}\delta[n] - \frac{1}{2}\delta[n-1]
\end{equation}

Y las respuestas en frecuencia de los mismos es:
\begin{equation*}
  H(e^{j2{\pi}s}) = \frac{TFTD\{y[n]\}}{TFTD\{x[n]\}} = \frac{Y(e^{j2{\pi}s})}{X(e^{j2{\pi}s})}
\end{equation*}
\begin{equation} \label{H1}
  H_1(e^{j2{\pi}s}) = \frac{1}{2} + \frac{1}{2}e^{-j2{\pi}s}
\end{equation}
\begin{equation} \label{H2}
  H_2(e^{j2{\pi}s}) = \frac{1}{2} - \frac{1}{2}e^{-j2{\pi}s}
\end{equation}

Para poder conseguir las respuestas impulsionales de los sistemas \ref{eq3} y \ref{eq4} cambia el procedimiento, porque hay una realimentaci\'on de la salida ($y[n]$), entonces se obtiene de la siguiente manera calculando la TFTD del sistema \ref{eq3}.
% Despeje de H = Y/X
\[ TFTD\{y[n] = \frac{1}{2}x[n] + \frac{1}{2}x[n-1] + \frac{1}{2}y[n-1]\} \]
\begin{equation*}
  Y(e^{j2{\pi}s}) = \frac{1}{4}X(e^{j2{\pi}s}) + \frac{1}{4}e^{-j2{\pi}s}X(e^{j2{\pi}s}) + \frac{1}{2}e^{-j2{\pi}s}Y(e^{j2{\pi}s})
\end{equation*}
\begin{equation*}
  Y(e^{j2{\pi}s}) [ 1 - \frac{1}{2}e^{-j2{\pi}s}] = X(e^{j2{\pi}s}) [ \frac{1}{4} + \frac{1}{4}e^{-j2{\pi}s}]
\end{equation*}
\begin{equation} \label{H3}
  H_3(e^{j2{\pi}s}) = \frac{Y(e^{j2{\pi}s})}{X(e^{j2{\pi}s})} = \frac{[ \frac{1}{4} + \frac{1}{4}e^{-j2{\pi}s}]}{[ 1 - \frac{1}{2}e^{-j2{\pi}s}]}
\end{equation}

Entonces a partir de la siguiente transformada
\begin{equation*}
  u[n]a^n \supset \frac{1}{1-ae^{-j2{\pi}s}} , |a|<1
\end{equation*}

y de la propiedad
\begin{equation*}
  x[n-n_o] \supset e^{-j2{\pi}sn_o}X(e^{-j2{\pi}s})
\end{equation*}

Se antitransforma la respuesta en frecuencia del sistema \ref{eq3} (ec. \ref{H3}) para obtener la respuesta impulsional del mismo (ec. \ref{h3})
\begin{equation*}
  H_3(e^{j2{\pi}s}) = \frac{1}{4}[\frac{1}{1 - \frac{1}{2}e^{-j2{\pi}s}} + \frac{e^{-j2{\pi}s}}{1 - \frac{1}{2}e^{-j2{\pi}s}}]
\end{equation*}
\begin{equation} \label{h3}
  H_3(e^{j2{\pi}s}) \supset h_3[n] = \frac{1}{4} [ u[n](\frac{1}{2})^n + u[n-1](\frac{1}{2})^{n-1} ]
\end{equation}

Ahora se procede de la misma manera con el sistema \ref{eq4}, y as\'i se obtienen la respuesta en frecuencia (ec. \ref{H4}) y la respuesta impulsional (ec. \ref{h4}) del mismo.
\begin{equation} \label{H4}
  H_4(e^{j2{\pi}s}) = \frac{1}{4}[ \frac{1}{1+\frac{1}{2}e^{-j2{\pi}s}} - \frac{e^{-j2{\pi}s}}{1+\frac{1}{2}e^{-j2{\pi}s}} ]
\end{equation}
\begin{equation*}
  H_4(e^{j2{\pi}s}) \supset h_4[n]
\end{equation*}
\begin{equation} \label{h4}
  h_4[n] = \frac{1}{4} [ u[n](\frac{-1}{2})^n - u[n-1](\frac{-1}{2})^{n-1} ]
\end{equation}

\subsection{Conclusiones sobre los sistemas} \label{subsec:conclusiones-sistemas}
Observando las respuestas impulsionales podemos concluir que corresponden a sistemas del tipo FIR\footnote{Finite Impulse Response (Respuesta Impulsional Finita).}. \\
Adem\'as, analizando las respuestas en frecuencias, el \textbf{sistema 1} (Fig. \ref{fig:resp_sist_1}) se comporta como un filtro pasa bajos, el \textbf{sistema 2} (Fig. \ref{fig:resp_sist_2}) como uno pasa altos, el \textbf{3} (Fig. \ref{fig:resp_sist_3}) como un pasa bajos con un ancho de banda m\'as selectivo que el sistema 1, y por \'ultimo el \textbf{sistema 4} (Fig. \ref{fig:resp_sist_4}) como un filtro pasa altos con una banda m\'as selectiva que el sistema 3.
\image{resp_sist_1}{Respuesta impulsional y en frecuencia del sistema 1.}
\image{resp_sist_2}{Respuesta impulsional y en frecuencia del sistema 2.}
\image{resp_sist_3}{Respuesta impulsional y en frecuencia del sistema 3.}
\image{resp_sist_4}{Respuesta impulsional y en frecuencia del sistema 4.}

\section{C\'alculo num\'erico de las respuestas impulsionales y en frecuencias de los sistemas}
A partir de las ecuaciones de los sistemas (\ref{eq1}, \ref{eq2}, \ref{eq3} y \ref{eq4}) se obtuvieron, en forma num\'erica, las respuestas impusionales, tomando como entrada a los sistemas una delta ($\delta[n]$); y tambi\'en la respuesta en frecuencia a partir de usar el algoritmo de la \textbf{TFTD} (ver \ref{tftd_function}) en la respuesta impulsional obtenida previamente.

\subsection{Implementaci\'on de los sistemas en Octave}
Para poder implementar cada sistema se codificaron funciones que describen a cada uno, luego se defini\'o una funci\'on an\'onima delta que representa una delta de kronocker ($\delta[n]$) y, por \'ultimo, un vector N que define el instante de la señal

\subsection{C\'odigo}
\begin{lstlisting}[style=Matlab-editor]
y1 = @(x, n) (.5)*x(n) + (.5)*x(n-1);
y2 = @(x, n) (.5)*x(n) - (.5)*x(n-1);
function [y] = Y3(x, n)
  y(1) = 0; % Condicion inicial.
  for (k = 2:length(n))
    y(k) = (.25).*x(n(k)) + (.25).*x(n(k-1)) + (.5).*y(k-1);
  end
end
y3 = @(x, n) Y3(x, n);
function [y] = Y4(x, n)
  y(1) = 0; % Condicion inicial.
  for (k = 2:length(n))
    y(k) = (.25).*x(n(k)) - (.25).*x(n(k-1)) - (.5).*y(k-1);
  end
end
y4 = @(x, n) Y4(x, n);

% Defino un impulso.
delta = @(n) (n==0);

N = [-20:20];

\end{lstlisting}

\subsection{Conclusiones sobre los resultados num\'ericos}
Observando los resultados que se pueden apreciar en los siguiente gr\'aficos (Fig. \ref{fig:num_resp_sist_1}, \ref{fig:num_resp_sist_2}, \ref{fig:num_resp_sist_3} y \ref{fig:num_resp_sist_4}) se concluye que \textit{las implementaci\'ones num\'ericas son tan buenas como las anal\'iticas} y, adem\'as, indican que los c\'alculos anal\'iticos son correctos.
\image{num_resp_sist_1}{Respuesta impulsional y en frecuencia del sistema 1 (num\'erica).}
\image{num_resp_sist_2}{Respuesta impulsional y en frecuencia del sistema 2 (num\'erica).}
\image{num_resp_sist_3}{Respuesta impulsional y en frecuencia del sistema 3 (num\'erica).}
\image{num_resp_sist_4}{Respuesta impulsional y en frecuencia del sistema 4 (num\'erica).}

\section{Procesamiento de \textit{``senial''} con los sistemas}
Se emplean los sistemas \ref{eq1}, \ref{eq2}, \ref{eq3} y \ref{eq4} como filtros para procesar la se\~nal dada \textit{``senial''}; para ello se usa como entrada a los mismos \textit{``senial''}; luego de esto se calculan las \textbf{TFTD} de las se\~nales filtradas. Los resultados obtenidos se pueden observar en los gr\'aficos \ref{fig:senial_filtrada_1}, \ref{fig:senial_filtrada_2}, \ref{fig:senial_filtrada_3} y \ref{fig:senial_filtrada_4}, donde \textit{``$senial_i$''} refiere a la señal  filtrada con el sistema \textit{``i''}.
\image{senial_filtrada_1}{Senial filtrada con el sistema 1.}
\image{senial_filtrada_2}{Senial filtrada con el sistema 2.}
\image{senial_filtrada_3}{Senial filtrada con el sistema 3.}
\image{senial_filtrada_4}{Senial filtrada con el sistema 4.}
\subsection{Conclusiones de las se\~nales filtradas}
A partir de los gr\'aficos \ref{fig:senial_filtrada_1}, \ref{fig:senial_filtrada_2}, \ref{fig:senial_filtrada_3} y \ref{fig:senial_filtrada_4} se puede concluir que las hip\'otesis planteadas en \ref{subsec:conclusiones-sistemas} son correctas, pues en los mismos se puede observar que las componentes de alta frecuencia son eliminadas con los sistemas \ref{eq1} y \ref{eq3} (Fig. \ref{fig:senial_filtrada_1} y \ref{fig:senial_filtrada_3} respectivamente), que se comportaban como \textbf{filtros pasa bajos}; y de manera similar las frecuencias bajas son eliminadas con los sistemas \ref{eq2} y \ref{eq4} (Fig. \ref{fig:senial_filtrada_2} y \ref{fig:senial_filtrada_4} respectivamente), que act\'uaban como \textbf{filtros pasa altos}

\section{Procesamiento de la se\~nal provista \textit{``hcanald''}}
La se\~nal provista \textit{``hcanald''} se compone de tres deltas de Kronocker de diferentes amplitudes, una ubicada en 0 y de amplitud $1$ ($\delta[n]$), otra en $n_0=8820$ y de amplitud $\alpha=\frac{2}{5}$ ($\alpha\delta[n-n_0]$), y la tercera en $n_1=2n_0=17640$ y de amplitud $\beta=\alpha^2=\frac{4}{25}$ ($\beta\delta[n-n_1]$). Por lo que la respuesta impulsional puede ser descripta anal\'iticamente como:
\begin{equation} \label{eq:hcanald}
  h[n] = \delta[n] + \alpha\delta[n-n_0] + \beta\delta[n-n_1]
\end{equation}
\image{hcanald}{Se\~nal \textit{``hcanald''}.}
\image{hcanald_analitica}{Se\~nal \ref{eq:hcanald}.}

Y como se puede observar en los gr\'aficos (Fig. \ref{fig:hcanald} y Fig. \ref{fig:hcanald_analitica}), la ecuaci\'on \ref{eq:hcanald} es correcta.

\subsection{Salida del canal digital}
A partir de la ecuaci\'on \ref{eq:hcanald} implementada como un sistema en Octave (ver \ref{code:ecdiff}) y de la se\~nal de audio provista, se obtuvo la salida de lo que simula ser un canal digital (Fig. \ref{fig:salidaCanalDigital}). 
\image{salidaCanalDigital}{Audio provisto como entrada y salida al canal digital.}

\subsection{Implementaci\'on del canal digital en Octave}
\label{code:ecdiff}
\begin{lstlisting}[style=Matlab-editor]
function y = ecDiff(x)
  % El algoritmo solo acepta que x sea vector.
  if (
      strcmp(typeinfo(x), 'bool matrix') == 0 &&
      strcmp(typeinfo(x), 'matrix') == 0
  )
    dispc('ERROR: x debe ser vector.\n', 'red');
    exit(1);
  end

  %{
    La ecuacion a resolver es:
    y[n] = x[n] + a.x[n-n0] + b.x[n-n1]
  %}

  % Constantes.
  n0 = 8820;
  n1 = 2*n0;
  a = (2/5);
  b = a^2;
  
  % Algoritmo.
  for (k = 1:length(x))
    term1=0;term2=0;
    if (k-n0 > 0)
      term1 = a*x(k-n0);
    end
    if (k-n1 > 0)
      term2 = b*x(k-n1);
    end
    y(k) = x(k) + term1 + term2;
  end

end
\end{lstlisting}

\subsection{Conclusiones sobre el canal digital}
De los resultados previos (Fig. \ref{fig:salidaCanalDigital}) el lector puede inferir correctamente (de manera visual y auditiva), que el canal digital distorsiona la se\~nal de entrada creando dos r\'eplicas de \textit{amplitudes significativas}; y estas son debidas a las deltas descriptas anteriormente (ver ec. \ref{eq:hcanald}). Para poder solucionar estos efectos indeseados se procede aplicando un filtro del tipo FIR, que busca asemejar la respuesta impulsional del canal a una \'unica delta de Kronecker ($h[n]=\delta[n]$), pues de esta manera las r\'eplicas no aparecen, ya que las deltas $\alpha\delta[n-n_0]$ y $\beta\delta[n-n_1]$ son la causa de las mismas. \\
El problema que ocurre al aplicar el filtro, es que siempre permanece una delta indeseada, cuya amplitud es proporcional a la cantidad de ramas del filtro. Esto se puede ver claramente en las siguientes ecuaciones, donde se calculan las repuestas impulsionales de los sistemas equivalentes en cascada $h_{fi}[n]$, donde $i$ corresponde a la cantidad de ramas del filtro. \\
Si suponemos la respuesta impulsional del canal como $h_c[n]$
\begin{equation*}
  h_c[n] = \delta[n] + \alpha\delta[n-n_0]
\end{equation*}
Entonces el sistema equivalente tiene una respuesta impulsional $h_{eqi}[n]$:
\begin{equation*}
  h_{eqi}[n] = conv\{ h_c[n] * h_{fi}[n] \}[n]
\end{equation*}
\begin{equation*}
  h_{f1}[n] = \delta[n] - \alpha\delta[n-n_0]
\end{equation*}
\begin{equation*}
  h_{eq1}[n] = \delta[n] - (\alpha)^2\delta[n-n_0]
\end{equation*}
Para una rama se tiene una delta indeseada de amplitud $\alpha^2$
\begin{equation*}
  h_{f2}[n] = \delta[n] - \alpha\delta[n-n_0] + (\alpha)^2\delta[n-2n_0]
\end{equation*}
\begin{equation*}
  h_{eq2}[n] = \delta[n] + (\alpha)^3\delta[n-3n_0]
\end{equation*}
Para dos ramas se tiene una delta indeseada de amplitud $\alpha^3$
\begin{equation*}
  h_{f3}[n] = \delta[n] - \alpha\delta[n-n_0] + (\alpha)^2\delta[n-2n_0] - (\alpha)^3\delta[n-3n_0]
\end{equation*}
\begin{equation*}
  h_{eq3}[n] = \delta[n] - (\alpha)^4\delta[n-4n_0]
\end{equation*}
Para tres ramas se tiene una delta indeseada de amplitud $\alpha^4$. Y as\'i sucesivamente, concluyendo que siempre se tiene un eco.

Por lo que para solucionar los ecos del canal digital, se comienza implementando un filtro de dos ramas ``$f1$'', $f_1[n]=x[n]-(\alpha)x[n-n_0]+(\alpha)^2x[n-2n_0]$, con $n_0=8820$ y $\alpha=\frac{2}{5}$, el resultado se puede ver en la figura \ref{fig:y-filtro1}.
\image{h-filtro1}{Respuesta impulsional del filtro 1.}
\image{y-filtro1}{Salida del filtro 1.}

Por lo explicado previamente la salida del filtro ``$f1$'' todav\'ia posee un poco de distorsi\'on, lo cual se puede mejorar implementando un filtro ``$f2$'' que posee tres ramas de retardo: $f_2[n] = x[n] - (\alpha)x[n-n_0] + (\alpha)^2x[n-2n_0] + (\alpha)^3x[n-3n_0]$.
\image{h-filtro2}{Respuesta impulsional del filtro 2.}
\image{y-filtro2}{Salida del filtro 2.}

\begin{minipage}{\linewidth}
  El lector puede corroborar, auditivamente u observando los gr\'aficos provistos que, la salida del filtro 2, es mucho mejor que la salida del canal digital, pero \textit{no id\'entica a la se\~nal original} antes de introducirla en el canal digital; por lo que se concluye que \textit{se pudieron mejorar}, a un nivel aceptable, las distorsiones que produce el canal; pero que tambi\'en, se podr\'ia seguir mejorando a\'un m\'as, \textit{implementando filtros con m\'as ramas}.
\end{minipage}

\section{Aclaraciones sobre datos ajenos al presente}
Todos los archivos y datos a los que refieren en el presente se pueden encontrar en las carpetas adjuntas al mismo; algunas de \'estas carpetas se generar\'an despu\'es de ejecutar \textit{``main.m''} con \textbf{Octave} (\textit{no se recomienda usar Matlab\footnote{MATLAB (abreviatura de MATrix LABoratory, ``laboratorio de matrices'') es un sistema de c\'omputo num\'erico que ofrece un entorno de desarrollo integrado (IDE) con un lenguaje de programaci\'on propio (lenguaje M).}, pues no ha sido dise\~nado para el mismo}).

\begin{thebibliography}{99}
  
\bibitem{bib:oppenheim}A.V. Oppenheim, A.S. Willsky, Se\~nales y Sistemas.
\bibitem{bib:tftd_tp5}ANÁLISIS DE SISTEMAS Y SE\~{n}ALES - A\~{n}O 2023, Práctica 5 Transformada de Fourier de Tiempo Discreto (TFTD), Serie Discreta Fourier (SDF).
\bibitem{bib:tftd_teoria}ANÁLISIS DE SISTEMAS Y SE\~{n}ALES - A\~{n}O 2023, Transparencias de teor\'ia 5: Transformada de Fourier de Tiempo Discreto (TFTD).

\end{thebibliography}

\end{document}
